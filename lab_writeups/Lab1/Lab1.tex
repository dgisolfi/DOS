\documentclass[a4paper, 11pt]{article}
\usepackage{comment} % enables the use of multi-line comments (\ifx \fi)  
\usepackage{fullpage} % changes the margin

\begin{document}
%Header-Make sure you update this information!!!!
\noindent
\large\textbf{Post/Pre-Lab 1 Report} \hfill \textbf{Daniel Gisolfi} \\
Prof. Labouseur \hfill Lab Date: 2018/09/11\\

\section*{Post Lab Questions}

\begin{enumerate}
\item What are the advantages and disadvantages of using the same system call interface 
for manipulating both files and devices? 
\item Would it be possible for the user to develop a new command interpreter using the 
system call interface provide by the operating system? How?
\end{enumerate}

\section*{Question 1}

\underline{Advantages}
\begin{itemize}
\item All devices added can be accessed as if it is just another file within the system.
\item Creates an interface to add new devices as drivers
\item Creates a layer of Abstraction for developer/programing 
\end{itemize}
\underline{Disadvantages}
\begin{itemize}
\item Performance can become an issue when many files and devices are attempting to make system calls concurrently.
\item Using file context may also result in a loss of functionality
\end{itemize}



\section*{Question 2}

Yes, a user should be able to develop a new command interpreter using the system call interface.
This can be done by creating user level programs that access system calls. 
Using the functionality found in the operating system it would be possible for a user to create a new CLI.

\end{document}
